\chapter{Theoretische Grundlagen}

\section{Agile Softwareentwicklung}
Agile Softwareentwicklung soll die Zusammenarbeit zwischen Entwicklern, Kunden und anderen Stakeholdern verbessern. Anstelle 
eines starren und vorhersehbaren Vorgehens betont die agile Methode die kontinuierliche Weiterentwicklung, iterative Prozesse 
und die enge Zusammenarbeit mit den Kunden.

Ein Framework für die Umsetzung von agiler Softwareentwicklung ist Scrum. Es basiert auf der Idee der Selbstorganisation und ermöglicht es Teams, sich auf die Wertsteigerung für den Kunden zu konzentrieren. Scrum besteht aus verschiedenen Rollen, Artefakten und Meetings.


\section{Modulare und containerisierte Funktionen}
Die rasante Entwicklung von Softwareanwendungen erfordert neue Ansätze zur Steigerung der Effizienz und Flexibilität in der Softwareentwicklung. 
In diesem Kontext sind modulare, containerisierte Softwareapplikationen entstanden, um diese Herausforderungen anzugehen. 
Diese Technologie ermöglicht es Entwicklern, Software in unabhängige Module aufzuteilen und in Containern zu isolieren, um 
eine bessere Skalierbarkeit, Portabilität und Wiederverwendbarkeit zu erreichen.

Eine der bekanntesten und am weitesten verbreiteten Technologien für die Containerisierung in modularen Softwareapplikationen ist Docker. Docker bietet eine Plattform für die Erstellung, Bereitstellung und Ausführung von Containern. Es erleichtert die Verpackung von Anwendungen und deren Abhängigkeiten in standardisierte Container, die auf verschiedenen Betriebssystemen und Infrastrukturen ausgeführt werden können.
Die Integration von Docker Containern in modulare Softwareapplikationen bringt zusätzliche Vorteile mit sich. Erstens ermöglicht Docker eine konsistente und reproduzierbare Umgebung für die Ausführung von Modulen, unabhängig von der zugrunde liegenden Infrastruktur. Entwickler können Container lokal erstellen und testen, bevor sie sie auf andere Umgebungen wie Produktionsserver oder Cloud-Plattformen bereitstellen. Dies erleichtert die Zusammenarbeit und den Austausch von Modulen zwischen verschiedenen Entwicklern und Teams.
Die Bereitstellung der Anwendung für den Kunden wird erheblich vereinfacht. Der Kunde muss keine komplexen Installationsschritte oder Konfigurationen durchführen, sondern kann den Container einfach auf seinem System ausführen.

Modulare, containerisierte Softwareapplikationen in Verbindung mit Docker bieten eine innovative Lösung für die effiziente Softwareentwicklung. Die Aufteilung von Anwendungen in unabhängige Module und die Isolierung dieser Module in Containern verbessern die Skalierbarkeit, Portabilität und Wiederverwendbarkeit von Software. Docker erleichtert die Erstellung, Bereitstellung und Verwaltung von Containern und ermöglicht eine konsistente Umgebung für die Ausführung von Modulen. 