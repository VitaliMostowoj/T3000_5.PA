\tikzmath{
    \winkel = 83;
    \hoehe = 1cm;
    \TanValue = {tan(\winkel)};
    function oberseite(\a) {
        return (\a - 2*(\hoehe / \TanValue));
    };
    \unten = 2cm;
    \mitte = oberseite(\unten);
    \oben = oberseite(\mitte);
    \obenoben = oberseite(\oben);
}
\tikzset{
    trapez/.style n args={1}{
        trapezium,
        thick,
        inner ysep = \hoehe,
        trapezium stretches = true,
        trapezium angle = \winkel,
        align = center,
    }
}

\begin{figure}
    \centering
    
    \begin{figure}
            \centering
            \begin{tikzpicture}
                \draw[step=1cm,gray,very thin] (-6cm,-6cm) grid (6cm,6cm);
                
                \node[trapez={\oben},draw,inner xsep = \obenoben] at (-2cm,2cm) {};
                \node[trapez={\mitte},draw,inner xsep = \oben] at (-2cm,0cm) {};
                \node[trapez={\unten},draw, inner xsep = \mitte] at (-2cm,-2cm) {};
                
                \node[trapez={\oben},inner xsep = \obenoben] at (-2cm,2cm) {Service-\\roboter};
                \node[trapez={\mitte},inner xsep = \oben] at (-2cm,0cm) {Mobiler\\Roboter};
                \node[trapez={\unten},inner xsep = \mitte] at (-2cm,-2cm) {Industrie-\\roboter};
                
                
                \node[trapez={\unten},draw, inner xsep = \mitte,rotate = 180] at (2cm,2cm) {};
                \node[trapez={\mitte},draw,inner xsep = \oben,rotate = 180] at (2cm,0cm) {};
                \node[trapez={\oben},draw,inner xsep = \obenoben,rotate = 180] at (2cm,-2cm) {};
        
        
                
                \draw[thick,->] (0,-4)--(0,4) node [anchor = south east] {y-Achse};     % erstellen eines Pfeiles in vertikaler Richtung mit Beschriftung
                \draw[thick,->](-4,0)--(4,0) node [anchor = north west] {x-Achse};      % erstellen eines Pfeiles in horizontaler Richtung mit Beschriftung
                
                
                \draw[red,thick](0,0) circle (3cm);
                
                \draw[blue,thick](-3,-3) -- (3,3);
                \draw[blue,thick](-3,3) -- (3,-3);
                
                \draw[green,thick](0,0) parabola (3,3);
                \draw[green,thick,rotate= 180](0,0) parabola (3,3);
                
                \node[trapez, red, minimum width = 3cm, minimum height = 2cm, trapezium left angle = 60, trapezium right angle = 120, inner sep = 2cm] at (0,0){};
                \node[trapez,minimum height = 2cm, minimum width = 2.5cm, inner sep = 1cm, trapezium angle = 80.41] at (0,-2){}; 
                \node[trapez,minimum height = 2cm, minimum width = 2cm,   inner sep = 0.75cm, trapezium angle = 80.41] at (0,0){};
                
                \node[trapez,minimum width = 1.5cm, inner sep = 0.5cm] at (-1,2){};
                \node[trapez,minimum width = 2cm, inner sep = 0.75cm] at (-1,0){};
                \node[trapez,minimum width = 2.5cm, inner sep = 1cm] at (-1,-2){};
                
                \node[trapez,minimum width = 2.5cm, inner sep = 1cm, rotate = 180] at (1,2){};
                \node[trapez,minimum width = 2cm, inner sep = 0.75cm, rotate = 180] at (1,0){};
                \node[trapez,minimum width = 1.5cm, inner sep = 0.5cm, rotate = 180] at (1,-2){};
        
                \node[trapez, minimum width = 2.5cm, inner sep = 0.5cm, minimum height = 6cm,trapezium angle = 67]at (-1,0){};
            \end{tikzpicture}
            
            
            
            
    \begin{tikzpicture}
        \draw[step=1cm,gray,very thin] (-6cm,-6cm) grid (6cm,6cm);
        \draw[thick,->] (0,-4)--(0,4) node [anchor = south east] {y-Achse};     % erstellen eines Pfeiles in vertikaler Richtung mit Beschriftung
        \draw[thick,->](-4,0)--(4,0) node [anchor = north west] {x-Achse};      % erstellen eines Pfeiles in horizontaler Richtung mit Beschriftung
        
        
        \draw[red,thick](0,0) circle (3cm);
        
        \draw[blue,thick](-3,-3) -- (3,3);
        \draw[blue,thick](-3,3) -- (3,-3);

        \draw[green,thick](0,0) parabola (3,3);
        \draw[green,thick,rotate= 180](0,0) parabola (3,3);

        \node[trapez, red, minimum width = 3cm, minimum height = 2cm, trapezium left angle = 60, trapezium right angle = 120, inner sep = 2cm] at (0,0){};
        \node[trapez,minimum height = 2cm, minimum width = 2.5cm, inner sep = 1cm, trapezium angle = 80.41] at (0,-2){}; 
        \node[trapez,minimum height = 2cm, minimum width = 2cm,   inner sep = 0.75cm, trapezium angle = 80.41] at (0,0){};

        \node[trapez,minimum width = 1.5cm, inner sep = 0.5cm] at (-1,2){};
        \node[trapez,minimum width = 2cm, inner sep = 0.75cm] at (-1,0){};
        \node[trapez,minimum width = 2.5cm, inner sep = 1cm] at (-1,-2){};

        \node[trapez,minimum width = 2.5cm, inner sep = 1cm, rotate = 180] at (1,2){};
        \node[trapez,minimum width = 2cm, inner sep = 0.75cm, rotate = 180] at (1,0){};
        \node[trapez,minimum width = 1.5cm, inner sep = 0.5cm, rotate = 180] at (1,-2){};

        \node[trapez, minimum width = 2.5cm, inner sep = 0.5cm, minimum height = 6cm,trapezium angle = 67]at (-1,0){};
        \end{tikzpicture}


    
\end{figure}



\todo{winkel berechnen}