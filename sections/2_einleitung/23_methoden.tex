\chapter{Methoden}
Für die Umsetzung des Projektes kommen verschiedene wissenschaftliche Methodiken zum Einsatz. Diese Methodiken werden
zum Projektbeginn geplant, um eine strukturierte und zielgerichtete Umsetzung sicherzustellen. Im Fazit wird auf die 
Umsetzung und Anwendung der Methoden in Bezug auf das Projekt eingegangen. In der \autoref{tab:methoden} sind die 
geplanten Methoden aufgelistet.

\begin{table}[H]                                
    \centering
    \caption{Methodenübersicht}
    \begin{tabular}{|l|l|} \hline
        \textbf{Arbeitspaket}   & \textbf{Angewendete Methoden}                                         \\ \hline
        Kick-Off                & Literaturrecherche, Experteninterviews                                \\ \hline
        Einarbeitung            & Pair-Programming, Literaturrecherche, Experteninterviews              \\ \hline  
        Ist-Analyse             & Recherche interne Dokumente, Reverse Engineering, Experteninterviews  \\ \hline
        Modellentwicklung       & Literaturrecherche, Use-Case-Modellierung                             \\ \hline 
        Implementierung         & Agile Methoden (Scrum), Test Driven Development                       \\ \hline
        Testen                  & Unit Tests, Integrationstest                                          \\ \hline
        Dokumentation           & Protokollieren, kommentieren von Code                                 \\ \hline
    \end{tabular}
    \label{tab:methoden}
\end{table}

Innerhalb des Kick-Offs sollen die Methoden Experteninterviews und Literaturrecherche eingesetzt werden, um ein Verständnis für das Projekt zu generieren. Diese Methoden sind 
auch Teil der Einarbeitungsphase, da auch in dieser Phase weiteres Wissen zu dem Projekt 
aufzubauen. Des Weiteren soll hier die Methode "Pair-Programming" eingesetzt werden. Dabei
soll gemeinsam mit einem erfahrenen Entwickler Softwareprogramme geschrieben werden, um die 
Programmiersprache und das Programmierumfeld kennenzulernen. Für die Ist-Analyse wird eine
detaillierte Dokumenten- und Literaturrecherche benötigt, um die vorhandenen Funktionen und 
Variablen im System kennenzulernen und für das Projekt notwendige Schnittstellen zu finden. 
Dabei ist auch der Punkt Reverse Engineering sehr wichtig, da aus einem bereits bestehenden 
Projekt Informationen extrahier und auf das eigenen Projekt angewendet werden sollen. 
Die Modellentwicklung soll neben einer Literaturrecherche auch eine Use-Case-Modellierung
beinhalten, um den Anwendungsfall des Projektes zu definieren. Die Implementierung soll nach
dem internen Vorgehen der agilen Entwicklung mittels Scrum geschehen. Auf die agile Entwicklung
wird im folgenden Kapitel näher drauf eingegangen. Des weiteren soll Test Driven Development
zum Einsatz kommen. Für das Testen der Applikation sollen automatische Unit und Integrationstests geschrieben und für die Dokumentation der Code Kommentiert werden.