\chapter{Einleitung}
Dieser Praxisbericht behandelt die Tätigkeiten der fünften Praxisphase bei der \ac{ABB} AG in Mannheim aus dem Blickwinkel des wissenschaftlichen Arbeitens. 
Der Einsatzbereich ist das AssetInsight™ Team im Bereich Digital der \ac{PAEN}. AssetInsight ist eine Lösung von \ac{ABB} die das zentrale Condition Monitoring
von rotierenden Maschinen ermöglicht und eine Zusammenfassung der Daten auf einem Dashboard visualisiert. Das Team befasst sich mit dem Produktmanagement, der Entwicklung,
dem Projektmanagement und dem Vertrieb dieser Kundenlösung. Diese Praxisphase befasst sich primär mit der Einarbeitung in die internen Tools und Technologien
als Vorbereitung auf die Bachelorarbeit.


\section{Problemstellung und Ziel der Arbeit}
Für die Bearbeitung der Bachelorthesis \glqq Entwicklung eines Prototyp-Algorithmus zur Ermittlung des Schmierzustandes eines Kugellagers an rotierenden Maschinen\grqq{} ist 
zunächst eine Einarbeitung in die internen Systeme und die agile Softwareentwicklung notwendig. 
Dafür soll in dem Bearbeitungszeitraum ein Verständnis für die internen Systeme geschaffen werden. Dazu zählen die internen Methodiken in der Bearbeitung von Projekten, sowie die Tools die
zur agilen Softwareentwicklung genutzt werden.
Das abschließende Ziel der Arbeit ist das Entwerfen einer modularen und containerisierten Software Applikation für die Erkennung von Grenzwertüberschreitungen der Temperatur bei 
rotierenden Maschinen, als Vorbereitung auf das Entwickeln einer komplexeren Software Applikation für den ABB Smart Sensor. 

