\chapter{Arbeitsplan}
Für die theoretische Vorbereitung auf das Projekt ist zunächst die Einarbeitung in die Progammiersprache Python und die agile Softwarentwicklung notwendig. Dafür werden 
die Themen modulare Programmierung und Containerisierung betrachtet. Anschließend 
wird ein softwareseitiges Modell für die Grenzwertüberschreitung der Temperatur entwickelt und abschließend die Software Applikation implementiert, getestet und dokumentiert.

\newcommand{\besetzt}{\multicolumn{1}{>{\columncolor{darkred}}c|}{}}
\newcommand{\setrow}[1]{\gdef\currentrowstyle{#1}\ignorespaces}
\begin{table}[H]                               
    \centering
    \caption{Klemmenbezeichnung}
    \begin{tabular}{|l|c|c|c|c|c|c|c|c|c|c|c|c|} \hline
        \multicolumn{1}{|c|}{\textbf{Arbeitspaket}} & \textbf{1} & \textbf{2} & \textbf{3} & \textbf{4} & \textbf{5} & \textbf{6} & \textbf{7} & \textbf{8} & \textbf{9} & \textbf{10} & \textbf{11} & \textbf{12} \\ \hline   
        Ist-Analyse                        & \besetzt &   &   &   &   &   &   &   &   &    &    &    \\ \hline
    \end{tabular}
\end{table}

